\begin{center}
  \Large
  \textbf{KATA PENGANTAR}
\end{center}

\addcontentsline{toc}{chapter}{KATA PENGANTAR}

\vspace{2ex}

% Ubah paragraf-paragraf berikut dengan isi dari kata pengantar

Puji dan syukur kehadirat Tuhan Yang Maha Esa atas segala karunia-Nya, sehingga penulis dapat menyelesaikan penelitian dengan judul \textbf{\textit{Smart Whiteboard:} Pengenalan Teks Huruf Balok Menggunakan \textit{You Only Look Once (YOLO).}}

Penelitian ini disusun dalam rangka pemenuhan bidang riset di Departemen Teknik Komputer ITS, serta digunakan sebagai persyaratan menyelesaikan pendidikan S1. Dalam penulisan buku ini, penulis mengucapkan terima kasih kepada seluruh pihak yang terlibat secara langsung maupun tidak langsung dalam penelitian ini, khususnya kepada:

\begin{enumerate}[nolistsep]

  \item Keluarga, Ibu, Bapak dan Saudara semua yang telah memberikan dorongan spiritual dan material dalam penyelesaian penelitian ini.

  \item Bapak Dr. Supeno Mardi Susiki Nugroho, ST., MT. selaku Kepala Departemen Teknik Komputer, Fakultas Teknik Elektro dan Informatika Cerdas, Institut Teknologi Sepuluh Nopember.

  \item Dr. Eko Mulyanto Yuniarno, S.T., M.T., dan Reza Fuad Rachmadi, S.T., M.T., Ph.D., selaku dosen pembimbing yang telah memberikan dukungan, arahan, dan bimbingan khususnya selama proses pengerjaan penelitian ini.
  
  \item Bapak dan Ibu dosen pengajar serta staff departemen teknik komputer atas pengajaran, bimbingan, serta perhatian yang diberikan kepada penulis selama ini.
  
  \item Chaira dan Vidityar selaku rekan berkontribusi dalam event-event dan organisasi-organisasi yang diikuti bersama penulis.
  
  \item Teman-teman sesama YOLO yang sudah bersama-sama belajar metode YOLO sedari awal penelitian direncanakan.

  \item Seluruh teman-teman Cah News, Aliansi Sobat Wuhu, Asisten Laboratorium B401, Mahasiswa Teknik Komputer 2018 ITS, e58 ITS yang senantiasa memberikan dukungan dan motivasi dalam pelaksanaan tugas akhir ini.

\end{enumerate}

Kesempurnaan hanya milik Tuhan, dengan itu penulis memohon segenap kritik dan saran yang membangun. Harapannya semoga penelitian ini dapat memberikan manfaat bagi kita semua, amin.

\begin{flushright}
  \begin{tabular}[b]{r}
    % Ubah kalimat berikut dengan tempat, bulan, dan tahun penulisan
    Surabaya, Mei 2022\\
    \\
    \\
    \\
    \\
    % Ubah kalimat berikut dengan nama mahasiswa
    Rifqi Alhakim Hariyantoputera
  \end{tabular}
\end{flushright}
