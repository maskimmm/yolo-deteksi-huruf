\begin{center}
  \large\textbf{ABSTRACT}
\end{center}

\addcontentsline{toc}{chapter}{ABSTRACT}

\vspace{2ex}

\begingroup
  % Menghilangkan padding
  \setlength{\tabcolsep}{0pt}

  \noindent
  \begin{tabularx}{\textwidth}{l >{\centering}m{3em} X}
    % Ubah kalimat berikut dengan nama mahasiswa
    \emph{Name}     &:& Rifqi Alhakim Hariyantoputera \\

    % Ubah kalimat berikut dengan judul tugas akhir dalam Bahasa Inggris
    \emph{Title}    &:& \textit{Smart Whiteboard: Block Letter Recognition Using You Only Look Once (YOLO)} \\

    % Ubah kalimat-kalimat berikut dengan nama-nama dosen pembimbing
    \textit{Advisor}  &:& 1. Dr. Eko Mulyanto Yuniarno, S.T., M.T. \\
                      & & 2. Reza Fuad Rachmadi, S.T., M.T., Ph.D. \\
  \end{tabularx}
\endgroup

% Ubah paragraf berikut dengan abstrak dari tugas akhir dalam Bahasa Inggris
\textit{Communication is a process where a person or persons, groups, organizations, or communities create and use information to connect with the environment and other people. Writing is a form of communication. With writing, an idea can be poured and conveyed to other parties without having to be in the same place with other parties, and an idea can be immortalized. 
In order to understand a document, segmentation of image documents into a sentence form is an important step to understand a document. Unlike printed documents which already have a certain format in their writing, handwritten documents have very much variations. Technological development innovation applied to the blackboard is an effort to increase the role of technology in education sector. Based on this background, a method is needed to detect and classify block text using YOLO to be applied to the smart whiteboard. The model creation process is carried out using the YOLOv5 deep learning framework with YOLOv5n, YOLOv5s, and YOLOv5m variants. Model validation was carried out using the evaluation metric mean average precision (mAP), precision, recall, and confusion matrix. Model testing is carried out based on several scenarios, namely scenarios of variation of distance of image taken, variation of light intensity scenarios, and scenarios of model variants. From the implementation of this study, it was found that the higher (complex) the variant of the model used, the better the model results obtained, but the amount of time for the model to be built will also be longer.}

% Ubah kata-kata berikut dengan kata kunci dari tugas akhir dalam Bahasa Inggris
\textit{Keywords}: \textit{Deep Learning, CNN, YOLO, Smart Whiteboard, Object Detection}.
