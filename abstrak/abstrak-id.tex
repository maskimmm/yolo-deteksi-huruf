\begin{center}
  \large\textbf{ABSTRAK}
\end{center}

\addcontentsline{toc}{chapter}{ABSTRAK}

\vspace{2ex}

\begingroup
  % Menghilangkan padding
  \setlength{\tabcolsep}{0pt}

  \noindent
  \begin{tabularx}{\textwidth}{l >{\centering}m{2em} X}
    % Ubah kalimat berikut dengan nama mahasiswa
    Nama Mahasiswa    &:& Rifqi Alhakim Hariyantoputera \\

    % Ubah kalimat berikut dengan judul tugas akhir
    Judul Tugas Akhir &:&	Deteksi Teks Huruf Balok Pada Papan Tulis Menggunakan \emph{You Only Look Once (YOLO)} \\

    % Ubah kalimat-kalimat berikut dengan nama-nama dosen pembimbing
    Pembimbing        &:& 1. Dr. Eko Mulyanto Yuniarno, S.T., M.T. \\
                      & & 2. Reza Fuad Rachmadi, S.T., M.T., Ph.D. \\
  \end{tabularx}
\endgroup

% Ubah paragraf berikut dengan abstrak dari tugas akhir
Komunikasi merupakan suatu proses ketika seseorang atau beberapa orang, kelompok, organisasi, atau masyarakat menciptakan dan mengunakan informasi agar terhubung dengan lingkungan dan orang lain. Tulisan merupakan salah satu bentuk ragam komunikasi. Dengan adanya tulisan, suatu ide atau gagasan dapat dituangkan dan disampaikan kepada pihak lain tanpa harus berada ditempat yang sama dengan pihak lain, serta suatu ide ataupun gagasan dapat diabadikan. Dalam rangka memahami suatu dokumen, segmentasi dokumen gambar menjadi suatu bentuk kalimat merupakan langkah penting untuk memahami suatu dokumen. Tidak seperti dokumen cetak yang telah memiliki format tertentu dalam penulisannya, dokumen bertulisan tangan memiliki variasi yang sangat banyak. Inovasi pengembangan teknologi yang diterapkan pada papan tulis merupakan upaya dalam rangka meningkatkan peranan teknologi dalam Pendidikan. Berdasarkan latar belakang tersebut, diperlukan metode untuk mendeteksi dan klasifikasi teks huruf balok menggunakan YOLO untuk diterapkan pada alat \textit{smart whiteboard}. Proses pembuatan model dilakukan menggunakan \textit{deep learning framework} YOLOv5 dengan varian YOLOv5n, YOLOv5s, dan YOLOv5m. Validasi model dilakukan menggunakan \textit{evaluation metric \textnormal{yaitu} mean Average Precision (mAP), precision, recall, \textnormal{dan} confusion matrix.} Pengujian model dilakukan berdasarkan beberapa skenario yaitu skenario jarak pengambilan gambar, skenario intensitas cahaya, dan skenario jenis varian model. Dari pelaksanaan penelitian ini didapatkan hasil yaitu semakin tinggi (kompleks) varian model yang digunakan maka semakin bagus hasil model yang didapatkan, namun waktu dalam proses pembuatan model akan menjadi lebih Panjang.

% Ubah kata-kata berikut dengan kata kunci dari tugas akhir
Kata Kunci: \textit{Deep Learning, CNN, YOLO, Smart Whiteboard, Object Detection}
