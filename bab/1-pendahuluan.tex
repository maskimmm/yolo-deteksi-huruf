% not fixed

\chapter{PENDAHULUAN}
\label{chap:pendahuluan}

% Ubah bagian-bagian berikut dengan isi dari pendahuluan

\section{Latar Belakang}
\label{sec:latarbelakang}

Komunikasi, pada dasarnya merupakan aktivitas dasar manusia. Dengan adanya komunikasi, manusia dapat saling berhubungan dengan satu sama lain. Tulisan merupakan salah satu bentuk ragam komunikasi. Dengan adanya tulisan, suatu ide atau gagasan dapat dituangkan dan disampaikan kepada pembaca secara asinkronus serta dapat diabadikan. Secara umum, ragam tulisan dibagi menjadi 2 yaitu tulisan cetak pada dokumen dan tulisan tangan.\par
Segmentasi dokumen gambar menjadi suatu bentuk kalimat merupakan langkah penting untuk memahami suatu dokumen. Tidak seperti dokumen cetak, segmentasi pada dokumen bertulisan tangan masih merupakan suatu hal yang menantang karena memiliki ukuran spasi yang tidak menentu antar hurufnya serta memiliki variasi bentuk gaya tulisan \citep*{ryu2015word}. Tulisan cetak pada dokumen merupakan tulisan yang pengaturan dan gaya penulisannya diatur dan dikenali oleh program komputer.\par
\textit{Optical Character Recognition (OCR)} adalah proses konversi gambar huruf menjadi karakter ASCII yang dikenali oleh komputer. Walaupun diklaim memiliki akurasi hingga 99\%, \textit{Optical Character Recognition (OCR)} yang ada saat ini memiliki penurunan akurasi ketika dihadapkan kepada gambar dengan kualitas rendah seperti \textit{noise} gambar, kualitas cetakan rendah, karakter berdekatan \citep*{ImageMalu2001approachtch}, dan karakter dengan variasi yang tidak umum (tulisan tangan).\par
Papan tulis merupakan suatu media yang biasa digunakan untuk menuangkan tulisan, ide, ataupun gagasan utamanya dalam proses belajar dan mengajar. Seiring dengan perkembangan teknologi, berbagai teknologi \textit{Internet of Things} diterapkan pada papan tulis sehingga memiliki fitur tambahan yang dapat mempermudah proses belajar dan mengajar. Telah banyak teknologi yang dapat memproyeksikan gambar atau tulisan pada komputer ke papan tulis pintar \citep*{kellerman2018smart}. 
% Namun, belum ada teknologi yang mampu mengenali tulisan tangan pada papan tulis pintar.


\section{Permasalahan}
\label{sec:permasalahan}
Penerapan teknologi berbasis IoT untuk melakukan proses pengenalan huruf tulisan tangan pada papan tulis masih merupakan sebuah tantangan dikarenakan teknologi OCR itu sendiri masih kurang mendukung pengenalan tulisan tangan pada papan tulis. Oleh karena itu, diperlukannya suatu metode untuk mendeteksi dan klasifikasi teks huruf balok yang nantinya dapat diterapkan pada alat \textit{Smart Whiteboard.}

% Permasalahan yang didapat yaitu diperlukannya suatu metode untuk mendeteksi dan klasifikasi teks huruf balok untuk diterapkan pada alat \textit{Smart Whiteboard.} 

\section{Batasan Masalah}
\label{sec:batasanmasalah}

Batasan masalah dari permasalahan ini yaitu tulisan tangan yang digunakan yaitu ditulis pada media papan tulis dan terbatas pada huruf balok.

\section{Tujuan}
\label{sec:Tujuan}

Berdasarkan permasalahan yang telah disebutkan, didapatkan tujuan dari dibuatnya tugas akhir ini yaitu untuk membuat program komputer yang dapat melakukan pengenalan teks huruf balok sehingga dapat diimplementasikan pada alat \textit{Smart Whiteboard.}

\section{Manfaat}
\label{sec:manfaat}

Manfaat dari dibuatnya tugas akhir ini yaitu untuk mempermudah proses pengenalan teks huruf balok papan tulis pada alat \textit{smart whiteboard.}

% Format Buku TA baru, ga pake sistematika penulisan

% \section{Sistematika Penulisan}
% \label{sec:sistematikapenulisan}

% Laporan penelitian tugas akhir ini terbagi menjadi \lipsum[1][1-3] yaitu:

% \begin{enumerate}[nolistsep]

%   \item \textbf{BAB I Pendahuluan}

%   Bab ini berisi \lipsum[2][1-5]

%   \vspace{2ex}

%   \item \textbf{BAB II Tinjauan Pustaka}

%   Bab ini berisi \lipsum[3][1-5]

%   \vspace{2ex}

%   \item \textbf{BAB III Desain dan Implementasi Sistem}

%   Bab ini berisi \lipsum[4][1-5]

%   \vspace{2ex}

%   \item \textbf{BAB IV Pengujian dan Analisa}

%   Bab ini berisi \lipsum[5][1-5]

%   \vspace{2ex}

%   \item \textbf{BAB V Penutup}

%   Bab ini berisi \lipsum[6][1-5]

% \end{enumerate}
